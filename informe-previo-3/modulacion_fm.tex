\documentclass{article}

\title{Modulación FM}
\author{Salazar Roa, Davis Bremdow}
\date{ Julio 2024}
\begin{document}
	\maketitle
	\newpage
	\section{Definir la desviación de frecuencia en FM}
	La desviación de frecuencia o desviación de frecuencia pico $\Delta$f es un parámetro FM que al igual que el indice de modulación $\beta$ se definen para una modulación de ancha banda (WBFM) cuyo $\beta$ > 0.2
		
	\section{Definir el índice de Modulación en FM e indicar cual es el índice usado en FM comercial}
	
	\section{Cual es el ancho de banda usado en la FM en el Perú}
	
	\section{Qué relación tiene el voltaje (volumen) en la señal moduladora con el ancho de banda}
	
	\section{Como se genera la FM con un VCO (Oscilador Controlado por Voltaje)}
	
	\section{Según el PNAF cual es el rango de operación de la FM comercial}
\end{document}