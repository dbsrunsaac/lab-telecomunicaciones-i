\documentclass[]{article}

\title{Modulación FM}
\author{Salazar Roa, Davis Bremdow}
\date{ Julio 2024}

\begin{document}
	\maketitle
	\newpage
	
	\section{Definir la desviación de frecuencia en FM}
	La desviación de frecuencia o desviación de frecuencia pico $\Delta$f es un parámetro FM que indica la variación en la frecuencia de la portadora en torno a una frecuencia especifica o central cuando el angulo de esta señal se ve alterado por una señal de información o moduladora, además como la variación de la frecuencia cuenta con ciertos limites constantes este parámetro $\Delta$f se puede considerar como la oscilación en frecuencia de la portadora.
	
	Por otro lado la desviación de frecuencia es un parámetro que nos permite definir la constante de sensibilidad la cual nos indica la variación de frecuencia por unidad de voltio de la señal moduladora, siendo a su vez esta nuevo valor relevante para el calculo del indice de modulación o $\beta$ en la modulación FM la cual se define como: \\
	\begin{equation}
		\beta = \frac{\Delta f}{f_m}
	\end{equation}
	
	
	\section{Definir el índice de Modulación en FM e indicar cual es el índice usado en FM comercial}
	
	\section{Cual es el ancho de banda usado en la FM en el Perú}
	
	\section{Qué relación tiene el voltaje (volumen) en la señal moduladora con el ancho de banda}
	
	\section{Como se genera la FM con un VCO (Oscilador Controlado por Voltaje)}
	
	
	\section{Según el PNAF cual es el rango de operación de la FM comercial}
\end{document}